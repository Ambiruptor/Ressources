\documentclass[11pt,a4paper]{article}
\usepackage{fullpage}
\usepackage[T1]{fontenc}
\usepackage[utf8]{inputenc}
\usepackage{hyperref}
\usepackage{pgf}
\usepackage{tikz}
\usepackage[nottoc, notlof, notlot]{tocbibind}
\usepackage{float}

%\setlength{\parindent}{2em}

\title{\textbf{Ambiruptor}\\The Lexical Ambiguation
    Interruptor\\~\\Final Report}
 \author{Boumediene Brikci Sid\\
         Maria Boritchev\\
         Victor Hublitz\\
         Simon Mauras\\
         Pierre Ohlmann\\
         Ievgeniia Oshurko\\
         Samir Tendjaoui\\
         Thi Xuan Vu}

\begin{document}

\maketitle

\vspace{2cm}
\subsection*{Abstract} 

The main point of our project is to develop a word-sense disambiguation tool. Our aim is to be able, given a certain text, to determine the actual meaning of each ambiguous word. To this end we use Wikipedia, and more specifically its internal links, in order to produce an annotated corpus from which a machine learning framework is developed. At the present time, we are clear about our objectives and we have chosen the abstract design of our future code. Furthermore, the coding part is now well advanced as we achieve mining Wikipedia and have already implemented several feature extractors.

\newpage

\tableofcontents

\newpage

\section{Presentation}

Word Sense Disambiguation is a Natural Language Processing task that lies in the assignment of the appropriate meaning to a word according to a given context, and its separation from other possible meanings. Since the 1940s, this problem has proved its difficulty and the lack of database has forced people to label each word manually.

Nowadays, the Internet creates new possibilities to get sufficiently big databases and the use of new machine-learning methods has given more efficient results on this open problem.    

%\subsection{Word Sense Disambiguation}

\noindent There are several possible applications of Word Sense Disambiguation:
\begin{itemize}
	\item Machine Translation
	\item Information Retrieval
	\item Semantic Parsing
	\item Speech Synthesis and Recognition
\end{itemize}

\subsection{Ambiruptor project}

The main objective of the \textbf{Ambiruptor project} is to produce an efficient tool that gives the correct meaning of ambiguous words in a text. Our tool will be based on several supervised machine learning concepts. We use Wikipedia to build our learning corpus and to annotate it according to its internal links.

\noindent All the code we produce is under the \href{http://www.gnu.org/licenses/gpl-3.0.html}{GNU GPLv3} license.

\subsection{Work team}

Our team consists of 8 master students of the ENS of Lyon: Boumediene Brikci Sid, Maria Boritchev, Victor Hublitz, Simon Mauras, Pierre Ohlmann, Ievgeniia Oshurko, Samir Tendjaoui and Thi Xuan Vu. The coordinators of the project are Simon Mauras and Ievgeniia Oshurko.

\end{document}